\documentclass{beamer}

\usepackage[utf8]{inputenc}

\usepackage{xcolor}
\usepackage{hyperref}

\title[git]{git --- an introduction}
\subtitle{Learning to commit}
\author{LiTHe kod}
\date{2021}

\newcommand{\temporalgray}[2]{\temporal<#1>{\phantom{#2}}{#2}{\textcolor{gray}{#2}}}

\newcommand{\keyword}[1]{\hspace{-1.0em}\colorbox{black}{\textcolor{white}{\textbf{#1}}}}

\begin{document}

\frame{\titlepage}
\frame{\tableofcontents}

%TODO about us

\section{Getting started}

\begin{frame}[fragile]
  \begin{center}
  \Huge \alert{DON'T\\PANIC}
  \end{center}
\end{frame}

\begin{frame}[fragile]
  \frametitle{What even is git?}
\end{frame}

\begin{frame}[fragile]
  \frametitle{What even is git? \dots and why is it better than email?}
\end{frame}

\begin{frame}[fragile]
  \frametitle{Getting started -- Getting binaries}
  
  Windows: git BASH, \url{gitforwindows.org}\\
  MacOS: brew, \url{git-scm.com}\\
  Linux: Big-brain time\\

\end{frame}

\begin{frame}[fragile]
  \frametitle{Getting started -- Telling git who you are}

  Your name:\\
  \texttt{\$ git config --global user.name "kodapa"} \\[1em]

  Your email:\\
  \texttt{\$ git config --global user.email "kodapa@lithekod.se"} \\

\end{frame}

\begin{frame}[fragile]
  \frametitle{Getting started -- Getting your repository}
  \begin{tabular}{ll}
    \texttt{git clone} & \\
    \texttt{git init} & \\
  \end{tabular}
\end{frame}

\AtBeginSection[] % Do nothing for \section*
{
\begin{frame}<beamer>
\frametitle{Outline}
\tableofcontents[currentsection]
\end{frame}
}

\section{Basic commands}
\subsection{Getting to know the family}

% When/why to use them
% How to use them
% (Maybe) what you expect to happen

% Edvard
\begin{frame}[fragile]
  \frametitle{Getting to know the family}

  \begin{tabular}{ll}
    \texttt{git add} & I want to keep these! \\
    \texttt{git commit} & I did a thing! \\
    \texttt{git push} & I want to share this! \\ % mention pull
  \end{tabular}
  \\ [2.0em]

  \texttt{add}, \texttt{commit}, \texttt{push} and \texttt{pull}$^*$
  are the four horsemen of version control. The bread and butter.
  \\ [2.0em]

  \small $^*$\texttt{pull} will be discussed later.

\end{frame}

\begin{frame}
  \frametitle{Getting to know the family -- add}
  \keyword{When}\\
    You have made changes that you want to commit.
    Usually the first command in a series of commands.
  \\ [0.5em]

  \keyword{How}\\
  \hspace{-1.0em}
  \begin{tabular}{ll}
    \texttt{git add hello.txt} & Stage the specified file \\
    \texttt{git add *.py} & Stage all python-files \\
    \texttt{git add -i} & Interactive prompt for adding \\
    \texttt{git add -p} & Add a part of a file \\
  \end{tabular}
  \\ [0.5em]

  \keyword{Expect}\\
  Silence.
\end{frame}

\begin{frame}
  \frametitle{Getting to know the family -- commit}
  \keyword{When}\\
    You have added a \emph{relevant} group of changes. Making
    the code ready for a new change. Preferably as small as possible.
  \\ [0.5em]

  \keyword{How}\\
  \hspace{-1.0em}
  \begin{tabular}{ll}
    \texttt{git commit} & Commit the added file(s) \\
    \texttt{git commit --amend} & Update the previous commit \\
    \texttt{git commit -m "msg"} & Specify a message, skips the editor \\
  \end{tabular}
  \\ [0.5em]

  \keyword{Expect}\\
  A texteditor pops up, where you write a descriptive commit message. And
  output similar to: \\
  \texttt{[master 3b65386] A descriptive message} \\
  \texttt{1 file changed, 1 insertion(+)}

\end{frame}

\begin{frame}
  \frametitle{Getting to know the family -- push}
  \keyword{When}\\
    After you've made commits you are happy with, or if the computer
    is about to blow up. If you haven't pushed, no one has to know what
    happened here.\\
  \\ [0.5em]

  \keyword{Push}\\
  \hspace{-1.0em}
  \begin{tabular}{ll}
    \texttt{git push} & Tell the server about your changes \\
  \end{tabular}
  \\ [0.5em]

  \keyword{Expect}\\ [0.1em]
  \small
  \texttt{Enumerating objects: 1, done.}\\
  \texttt{Counting objects: 100\% (1/1), done.}\\
  \texttt{Writing objects: 100\% (1/1), 192 bytes | 192.00 KiB/s, done.}\\
  \texttt{Total 1 (delta 0), reused 0 (delta 0), pack-reused 0}\\
  \texttt{To github.com:kodapan/verkligt.git}\\
  \texttt{   192c869..2ab0e3c  main -> main}
\end{frame}

\subsection{Commands for information}

% Gustav
\begin{frame}[fragile]
  \frametitle{Commands for information}

  \begin{tabular}{ll}
    \texttt{git status} & What is happening? \\
    \texttt{git diff} & What has changed? \\
    \texttt{git log} & What has happened? \\
    \texttt{git show} & What changed over there? \\
  \end{tabular}

\end{frame}

\begin{frame}
  \frametitle{Commands for information -- status}
\end{frame}

\begin{frame}
  \frametitle{Commands for information -- diff}
\end{frame}

\begin{frame}
  \frametitle{Commands for information -- log}
\end{frame}

\begin{frame}
  \frametitle{Commands for information -- show}
\end{frame}

\subsection{Branching out}

% Erik
\begin{frame}[fragile]
  \frametitle{Branching out}

  \begin{tabular}{ll}
    \texttt{git checkout} & What is happening? \\
    \texttt{git branch} & What has happened? \\
    \texttt{git log} & What happened there? \\
  \end{tabular}

\end{frame}

\begin{frame}
  \frametitle{Branching out -- checkout}
\end{frame}

\begin{frame}
  \frametitle{Branching out -- branch}
\end{frame}

\begin{frame}
  \frametitle{Branching out -- log (again!)}
\end{frame}

\subsection{Branching in}

\begin{frame}[fragile]
  \frametitle{Branching in}

  \begin{tabular}{ll}
    \texttt{git merge} & I want both this and that! \\
    \texttt{git pull} & I want \emph{your} stuff! \\
  \end{tabular}

\end{frame}

\begin{frame}
  \frametitle{Branching in -- merge}
\end{frame}

\begin{frame}
  \frametitle{Branching in -- pull}
\end{frame}

%%% Working with others

% Pause? 5-10 minutes

\section{Working with others}
\subsection{Creating your first Merge Request}

\begin{frame}[fragile]
  \frametitle{Creating your first Merge Request}
\end{frame}

\begin{frame}[fragile]
  \frametitle{Creating your first Merge Request -- Creating your branch}
\end{frame}

\begin{frame}[fragile]
  \frametitle{Creating your first Merge Request -- Doing your thing}
\end{frame}

\begin{frame}[fragile]
  \frametitle{Creating your first Merge Request -- Pushing your branch}
\end{frame}

\begin{frame}[fragile]
  \frametitle{Creating your first Merge Request -- Asking for feedback}
\end{frame}

\subsection{Reading your first Merge Request}

\begin{frame}[fragile]
  \frametitle{Reading your first Merge Request}
\end{frame}

\begin{frame}[fragile]
  \frametitle{Reading your first Merge Request -- Getting their changes}
\end{frame}

\begin{frame}[fragile]
  \frametitle{Reading your first Merge Request -- Testing their changes}
\end{frame}

\begin{frame}[fragile]
  \frametitle{Reading your first Merge Request -- Reviewing their changes}  % maybe two slides
\end{frame}

\subsection{Project management}

\begin{frame}[fragile]
  \frametitle{Project management}
\end{frame}

\begin{frame}[fragile]
  \frametitle{Project management -- Commit messages}
\end{frame}

\begin{frame}[fragile]
  \frametitle{Project management -- Tags}
\end{frame}

\begin{frame}[fragile]
  \frametitle{Project management -- git $\sim$flow$\sim$}
\end{frame}

\begin{frame}[fragile]
  \frametitle{Project management -- CI} % not always needed etc
\end{frame}

\begin{frame}[fragile]
  \frametitle{Project management -- gitignore}
\end{frame}

\section{Final stretch}
\subsection{Advanced commands}

\begin{frame}[fragile]
  \frametitle{Some more commands (bonus round)}
  These commands can be scary and confusing, and hard to `undo'. Worth a google if you
  want to get more use out of git.
  \vspace{1em}
  
  \texttt{git worktree} -- Looking at other branches \\
  \texttt{git cherry-pick} -- Stealing commits \\
  \texttt{git stash} -- WIP commits \\
    
\end{frame}

\subsection{Finding help}

\begin{frame}[fragile]
  \frametitle{Finding help}

  Human contact \\
  \hspace{1em} LiTHe kod meetups -- tusedays 17:15-late,  \\
  \vspace{1em}

  In your terminal at your fingetips \\
  \hspace{1em} \texttt{\$ man git} -- Exhaustive, good if you know what you want \\
  \hspace{1em} \texttt{\$ man giteveryday} -- Short description of commands \\
  \hspace{1em} \texttt{\$ man gittutorial} -- A guide for getting started \\
  \hspace{1em} \texttt{\$ man git commit} -- Exhaustive information about the `commit' subcommand\\
  \vspace{1em}

  From the web, just a GET away \\
  \hspace{1em} \url{git-scm.com/doc} -- Links and reference manual \\
  \hspace{1em} \url{lithekod.se/gitcheatsheet/} -- LiTHe kods cheat-sheet \\
  \hspace{1em} \url{shorturl.at/jpCDQ} -- GitHub's cheat-sheet \\
  \hspace{1em} \url{shorturl.at/tuLO3} -- Flowchart by Justin Hileman for messes\\
  %TODO meetups

\end{frame}

\end{document}
