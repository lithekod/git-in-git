\documentclass{beamer}

\usepackage[utf8]{inputenc}

\usepackage{xcolor}
\usepackage{hyperref}

\title[git]{git --- an introduction}
\subtitle{Learning to commit}
\author{LiTHe kod}
\date{2021}

\newcommand{\temporalgray}[2]{\temporal<#1>{\phantom{#2}}{#2}{\textcolor{gray}{#2}}}

\begin{document}

\frame{\titlepage}

\section{Getting started}

\begin{frame}[fragile]
  \begin{center}
  \Huge \alert{DON'T\\PANIC}
  \end{center}
\end{frame}

\begin{frame}[fragile]
  \frametitle{Getting git -- Getting Binaries}
  
  Windows: git BASH, \url{gitforwindows.org}\\
  MacOS: brew, \url{git-scm.com}\\
  Linux: Big-brain time\\

\end{frame}

\begin{frame}[fragile]
  \frametitle{Getting git -- Getting Help}
  
  \vspace{1em}
  In your terminal at your fingetips \\
  \hspace{1em} \texttt{\$ man git} -- Exhaustive, good if you know what you want \\
  \hspace{1em} \texttt{\$ man giteveryday} -- Short description of commands \\
  \hspace{1em} \texttt{\$ man gittutorial} -- A guide for getting started \\
  \hspace{1em} \texttt{\$ man git commit} -- Exhaustive information about the `commit' subcommand\\
  \vspace{1em}
  From the web, just a GET away \\
  \hspace{1em} \url{git-scm.com/doc} -- Links and reference manual \\
  \hspace{1em} \url{lithekod.se/gitcheatsheet/} -- LiTHe kods cheat-sheet \\
  \hspace{1em} \url{shorturl.at/jpCDQ} -- GitHub's cheat-sheet \\
  \hspace{1em} \url{shorturl.at/tuLO3} -- Flowchart by Justin Hileman for messes\\
    
\end{frame}

\begin{frame}[fragile]
  \frametitle{Getting git -- Start a Config} 
  
  \texttt{\$ git config --global user.name `kodapa'} \\
  \texttt{\$ git config --global user.email `kodapa@lithekod.se'} \\
  Setting name and email. \\
  \vspace{1em}
  As a file: `$\sim$/.gitconfig'
    
\end{frame}

\begin{frame}[fragile]
  \frametitle{Getting git -- Example Config} 
  
  \begin{verbatim}
[user]
  email = edvard.thornros@gmail.com
  name = Edvard Thörnros
[log]
  abbrevCommit = true
  oneline = true
  graph = true
[add]
  verbose = true
[push]
  default = tracking
  followTags = true
[commit]
  verbose = true
[rerere]
  enable = true
  \end{verbatim}
    
\end{frame}

\begin{frame}[fragile]
  \frametitle{Getting to know the family}

  \temporalgray{+}{\texttt{\$ git add} -- I want to keep these} \\
  \temporalgray{+}{\texttt{\$ git commit} -- I did a thing!} \\
  \temporalgray{+}{\texttt{\$ git push} -- Tell the world what you've done} \\
  \temporalgray{+}{\texttt{\$ git checkout} -- Change where `here' is} \\
\end{frame}
  
\begin{frame}[fragile]
  \frametitle{Commands for information}

  \temporalgray{+}{\texttt{\$ git status} -- What is happening?} \\
  \temporalgray{+}{\texttt{\$ git log} -- What has happened?} \\
  \temporalgray{+}{\texttt{\$ git show} -- What happened there?} \\
\end{frame}

\begin{frame}[fragile]
  \frametitle{Some more commands (bonus round)}
  These commands can be scary and confusing, and hard to `undo'. Worth a google if you
  want to get more use out of git.
  \vspace{1em}
  
  \temporalgray{+}{\texttt{\$ git rebase} -- Change the order of commits} \\
  \temporalgray{+}{\texttt{\$ git worktree} -- All the files!} \\
  \temporalgray{+}{\texttt{\$ git cherry-pick} -- `rebase lite'} \\
    
\end{frame}

\end{document}
