\documentclass{beamer}

\usepackage[utf8]{inputenc}

\usepackage{fancyvrb}
\usepackage{hyperref}
\usepackage{xcolor}

\title[git]{git --- an introduction}
\subtitle{Learning to commit}
\author{LiTHe kod}
\date{2021}

\newcommand{\temporalgray}[2]{\temporal<#1>{\phantom{#2}}{#2}{\textcolor{gray}{#2}}}

\newcommand{\keyword}[1]{\hspace{-1.0em}\colorbox{black}{\textcolor{white}{\textbf{#1}}}}

\begin{document}

\frame{\titlepage}
\frame{\tableofcontents}

%TODO about us

\section{Getting started}

\begin{frame}[fragile]
  \begin{center}
  \Huge \alert{DON'T\\PANIC}
  \end{center}
\end{frame}

\begin{frame}[fragile]
  \frametitle{What even is git?}
\end{frame}

\begin{frame}[fragile]
  \frametitle{What even is git? \dots and why is it better than email?}
\end{frame}

\begin{frame}[fragile]
  \frametitle{Getting started -- Getting binaries}
  
  Windows: git BASH, \url{gitforwindows.org}\\
  MacOS: brew, \url{git-scm.com}\\
  Linux: Big-brain time\\

\end{frame}

\begin{frame}[fragile]
  \frametitle{Getting started -- Telling git who you are}

  Your name:\\
  \texttt{\$ git config --global user.name "kodapa"} \\[1em]

  Your email:\\
  \texttt{\$ git config --global user.email "kodapa@lithekod.se"} \\

\end{frame}

\begin{frame}[fragile]
  \frametitle{Getting started -- Getting your repository}
  \begin{tabular}{ll}
    \texttt{git clone} & \\
    \texttt{git init} & \\
  \end{tabular}
\end{frame}
%%\begin{frame}[fragile]
%%  \frametitle{Getting git -- Getting Help}
%%
%%  \vspace{1em}
%%  In your terminal at your fingetips \\
%%  \hspace{1em} \texttt{\$ man git} -- Exhaustive, good if you know what you want \\
%%  \hspace{1em} \texttt{\$ man giteveryday} -- Short description of commands \\
%%  \hspace{1em} \texttt{\$ man gittutorial} -- A guide for getting started \\
%%  \hspace{1em} \texttt{\$ man git commit} -- Exhaustive information about the `commit' subcommand\\
%%  \vspace{1em}
%%  From the web, just a GET away \\
%%  \hspace{1em} \url{git-scm.com/doc} -- Links and reference manual \\
%%  \hspace{1em} \url{lithekod.se/gitcheatsheet/} -- LiTHe kods cheat-sheet \\
%%  \hspace{1em} \url{shorturl.at/jpCDQ} -- GitHub's cheat-sheet \\
%%  \hspace{1em} \url{shorturl.at/tuLO3} -- Flowchart by Justin Hileman for messes\\
%%  %TODO meetups
%%
%%\end{frame}

\AtBeginSection[] % Do nothing for \section*
{
\begin{frame}<beamer>
\frametitle{Outline}
\tableofcontents[currentsection]
\end{frame}
}

\section{Basic commands}
\subsection{Getting to know the family}

% When/why to use them
% How to use them
% (Maybe) what you expect to happen

% Edvard
\begin{frame}[fragile]
  \frametitle{Getting to know the family}

  \begin{tabular}{ll}
    \texttt{git add} & I want to keep these! \\
    \texttt{git commit} & I did a thing! \\
    \texttt{git push} & I want to share this! \\ % mention pull
  \end{tabular}

\end{frame}

\begin{frame}
  \frametitle{Getting to know the family -- add}
\end{frame}

\begin{frame}
  \frametitle{Getting to know the family -- commit}
\end{frame}

\begin{frame}
  \frametitle{Getting to know the family -- push}
\end{frame}

\subsection{Commands for information}

% Gustav
\begin{frame}[fragile]
  \frametitle{Commands for information}

  \begin{tabular}{ll}
    \texttt{git status} & What is happening? \\
    \texttt{git diff} & What has changed? \\
    \texttt{git log} & What has happened? \\
    \texttt{git show} & What changed over there? \\
  \end{tabular}

\end{frame}

\begin{frame}
  \frametitle{Commands for information -- status}

  \keyword{When}\\
  Always! Guaranteed to not break anything.
  \\[0.5em]

  \keyword{How}\\
  \texttt{\$ git status}
  \\[0.5em]

  \keyword{Expect}\\
  Some text explaining what's currently happening.
\end{frame}

\begin{frame}[fragile]
  \frametitle{Commands for information -- status}

\begin{verbatim}
$ git status
\end{verbatim}
\pause{}
\begin{verbatim}
On branch master
nothing to commit, working tree clean
\end{verbatim}
\end{frame}

\begin{frame}[fragile]
  \frametitle{Commands for information -- status}

\begin{verbatim}
$ vim my_file.py
$ git status
\end{verbatim}
\pause{}
\vspace{-3.5ex}
\begin{verbatim}
On branch master
\end{verbatim}
\pause{}
\vspace{-3.5ex}
\begin{verbatim}
Changes not staged for commit:
\end{verbatim}
\pause{}
\vspace{-3.5ex}
\begin{verbatim}
  (use "git add <file>..." to update what will be
   commited)
\end{verbatim}
\pause{}
\vspace{-3.5ex}
\begin{verbatim}
  (use "git checkout -- <file>..." to discard changes
   in working directory)
\end{verbatim}
\pause{}
\vspace{-3.5ex}
\begin{verbatim}

        modified:   my_file.py
\end{verbatim}
\pause{}
\vspace{-3.5ex}
\begin{verbatim}

no changes added to commit (use "git add" and/or
                            "git commit -a")
\end{verbatim}
\end{frame}

\begin{frame}
  \frametitle{Commands for information -- diff}

  \keyword{When}\\
  You want to see the difference between two things. Very versatile.
  \\[0.5em]

  \keyword{How}\\
  \texttt{\$ git diff}
  \\[0.5em]

  \keyword{Expect}\\
  A so called \emph{diff}.
\end{frame}

\begin{frame}[fragile]
  \frametitle{Commands for information -- diff}

\begin{verbatim}
$ git diff
\end{verbatim}
\pause{}
\vspace{-2.4ex}
{
\color{gray}
\begin{Verbatim}[commandchars=\\\{\}]
diff --git a/my_file.py b/my_file.py
index a40ce22..ea7235e 100644
--- a/my_file.py
+++ b/my_file.py
\end{Verbatim}
}
\pause{}
\begin{Verbatim}[commandchars=\\\{\}]
@@ -1,5 +1,5 @@
 def main():
\textcolor{red}{-    print("Hello World")}
\textcolor{blue}{+    print("Hello World!")}
 
 if __name__ == "__main__":
     main()
\end{Verbatim}
\end{frame}

\begin{frame}
  \frametitle{Commands for information -- log}

  \keyword{When}\\
  You want to see an overview of what's changed over time.
  \\[0.5em]

  \keyword{How}\\
  \texttt{\$ git log}

  \texttt{\$ git log <branch>}
  \\[0.5em]

  \keyword{Expect}\\
  The log.
\end{frame}

\begin{frame}[fragile]
  \frametitle{Commands for information -- log}
\begin{Verbatim}
$ git add my_file.py
$ git commit
$ git log
\end{Verbatim}
\pause{}
\vspace{-1.2ex} % this spacing can heck off
\begin{Verbatim}
commit 4582c8c2c41d6c7c46bab45d1aae807afacc357e
Author: Kodapan <kodapan@lithekod.se>
Date:   Fri Jan 22 17:48:27 2021 +0100

    add proper punctutation

commit 8e94d9e8bb33496484c80eaff68541dc0b8cccbc
Author: Kodapan <kodapan@lithekod.se>
Date:   Fri Jan 22 17:43:14 2021 +0100

    my first commit
\end{Verbatim}
\end{frame}

\begin{frame}[fragile]
  \frametitle{Commands for information -- log \texttt{--}graph}
\begin{Verbatim}[commandchars=\\\{\}]


$ git log --graph
* commit 4582c8c2c41d6c7c46bab45d1aae807afacc357e
| Author: Kodapan <kodapan@lithekod.se>
| Date:   Fri Jan 22 17:48:27 2021 +0100
| 
|     add proper punctuation
| 
* commit 8e94d9e8bb33496484c80eaff68541dc0b8cccbc
  Author: Kodapan <kodapan@lithekod.se>
  Date:   Fri Jan 22 17:43:14 2021 +0100
  
      my first commit
\end{Verbatim}
\end{frame}

\begin{frame}
  \frametitle{Commands for information -- show}

  \keyword{When}\\
  You want more information about a specific commit.
  \\[0.5em]

  \keyword{How}\\
  \texttt{\$ git show <commit>}
  \\[0.5em]

  \keyword{Expect}\\
  Log message and diff output for the commit.
\end{frame}

\begin{frame}[fragile]
  \frametitle{Commands for information -- show}
\begin{Verbatim}[commandchars=\\\{\}]


$ git log --graph
* commit 4582c8c2c41d6c7c46bab45d1aae807afacc357e
| Author: Kodapan <kodapan@lithekod.se>
| Date:   Fri Jan 22 17:48:27 2021 +0100
| 
|     add proper punctuation
| 
* commit 8e94d9e8bb33496484c80eaff68541dc0b8cccbc
  Author: Kodapan <kodapan@lithekod.se>
  Date:   Fri Jan 22 17:43:14 2021 +0100
  
      my first commit
\end{Verbatim}
\end{frame}

\begin{frame}[fragile]
  \frametitle{Commands for information -- show}
\begin{Verbatim}[commandchars=\\\{\}]


$ git log --graph
* commit \textcolor{red}{4582}c8c2c41d6c7c46bab45d1aae807afacc357e
| Author: Kodapan <kodapan@lithekod.se>
| Date:   Fri Jan 22 17:48:27 2021 +0100
| 
|     add proper punctuation
| 
* commit 8e94d9e8bb33496484c80eaff68541dc0b8cccbc
  Author: Kodapan <kodapan@lithekod.se>
  Date:   Fri Jan 22 17:43:14 2021 +0100
  
      my first commit
\end{Verbatim}
\end{frame}

\begin{frame}[fragile]
  \frametitle{Commands for information -- show}
\begin{Verbatim}[commandchars=\\\{\}]
$ git show \textcolor{red}{4582}
\end{Verbatim}
\pause{}
\vspace{-2.5ex}
\begin{verbatim}
commit 4582c8c2c41d6c7c46bab45d1aae807afacc357e
Author: Kodapan <kodapan@lithekod.se>
Date:   Fri Jan 22 17:48:27 2021 +0100
\end{verbatim}
\vspace{-3ex}
\begin{verbatim}
    add proper punctutation
\end{verbatim}
\vspace{-2ex}
\begin{Verbatim}[commandchars=\\\{\}]
diff --git a/my_file.py b/my_file.py
index a40ce22..ea7235e 100644
--- a/my_file.py
+++ b/my_file.py
@@ -1,5 +1,5 @@
 def main():
\textcolor{red}{-    print("Hello World")}
\textcolor{blue}{+    print("Hello World!")}
...
\end{Verbatim}
\end{frame}

\subsection{Branching out}

% Erik
\begin{frame}[fragile]
  \frametitle{Branching out}

  \begin{tabular}{ll}
    \texttt{git checkout} & What is happening? \\
    \texttt{git branch} & What has happened? \\
    \texttt{git log} & What happened there? \\
  \end{tabular}

\end{frame}

\begin{frame}
  \frametitle{Branching out -- checkout}
\end{frame}

\begin{frame}
  \frametitle{Branching out -- branch}
\end{frame}

\begin{frame}
  \frametitle{Branching out -- log (again!)}
\end{frame}

\subsection{Branching in}

\begin{frame}[fragile]
  \frametitle{Branching in}

  \begin{tabular}{ll}
    \texttt{git merge} & I want both this and that! \\
    \texttt{git pull} & I want \emph{your} stuff! \\
  \end{tabular}

\end{frame}

\begin{frame}
  \frametitle{Branching in -- merge}
\end{frame}

\begin{frame}
  \frametitle{Branching in -- pull}
\end{frame}

%%% Working with others

% Pause? 5-10 minutes

\section{Working with others}
\subsection{Creating your first Merge Request}

\begin{frame}[fragile]
  \frametitle{Creating your first Merge Request}
\end{frame}

\begin{frame}[fragile]
  \frametitle{Creating your first Merge Request -- Creating your branch}
\end{frame}

\begin{frame}[fragile]
  \frametitle{Creating your first Merge Request -- Doing your thing}
\end{frame}

\begin{frame}[fragile]
  \frametitle{Creating your first Merge Request -- Pushing your branch}
\end{frame}

\begin{frame}[fragile]
  \frametitle{Creating your first Merge Request -- Asking for feedback}
\end{frame}

\subsection{Reading your first Merge Request}

\begin{frame}[fragile]
  \frametitle{Reading your first Merge Request}
\end{frame}

\begin{frame}[fragile]
  \frametitle{Reading your first Merge Request -- Getting their changes}
\end{frame}

\begin{frame}[fragile]
  \frametitle{Reading your first Merge Request -- Testing their changes}
\end{frame}

\begin{frame}[fragile]
  \frametitle{Reading your first Merge Request -- Reviewing their changes}  % maybe two slides
\end{frame}

\subsection{Project management}

\begin{frame}[fragile]
  \frametitle{Project management}
\end{frame}

\begin{frame}[fragile]
  \frametitle{Project management -- Commit messages}
\end{frame}

\begin{frame}[fragile]
  \frametitle{Project management -- Tags}
\end{frame}

\begin{frame}[fragile]
  \frametitle{Project management -- git $\sim$flow$\sim$}
\end{frame}

\begin{frame}[fragile]
  \frametitle{Project management -- CI} % not always needed etc
\end{frame}

\begin{frame}[fragile]
  \frametitle{Project management -- gitignore}
\end{frame}

\section{Advanced commands}

\begin{frame}[fragile]
  \frametitle{Some more commands (bonus round)}
  These commands can be scary and confusing, and hard to `undo'. Worth a google if you
  want to get more use out of git.
  \vspace{1em}
  
  \texttt{git worktree} -- Looking at other branches \\
  \texttt{git cherry-pick} -- Stealing commits \\
  \texttt{git stash} -- WIP commits \\
    
\end{frame}

\end{document}
