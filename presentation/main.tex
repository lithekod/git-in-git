\documentclass{beamer}

\usepackage[utf8]{inputenc}

\usepackage{xcolor}
\usepackage{hyperref}

\title[git]{git --- an introduction}
\subtitle{Learning to commit}
\author{LiTHe kod}
\date{2021}

\newcommand{\temporalgray}[2]{\temporal<#1>{\phantom{#2}}{#2}{\textcolor{gray}{#2}}}

\begin{document}

\frame{\titlepage}
\frame{\tableofcontents}

%TODO about us

\section{Getting started}

\begin{frame}[fragile]
  \begin{center}
  \Huge \alert{DON'T\\PANIC}
  \end{center}
\end{frame}

\begin{frame}[fragile]
  \frametitle{What even is git?}
\end{frame}

\begin{frame}[fragile]
  \frametitle{What even is git? \dots and why is it better than email?}
\end{frame}

\begin{frame}[fragile]
  \frametitle{Getting started -- Getting binaries}
  
  Windows: git BASH, \url{gitforwindows.org}\\
  MacOS: brew, \url{git-scm.com}\\
  Linux: Big-brain time\\

\end{frame}

\begin{frame}[fragile]
  \frametitle{Getting started -- Telling git who you are}

  Your name:\\
  \texttt{\$ git config --global user.name "kodapa"} \\[1em]

  Your email:\\
  \texttt{\$ git config --global user.email "kodapa@lithekod.se"} \\

\end{frame}

\begin{frame}[fragile]
  \frametitle{Getting started -- Getting your repository}
  \begin{tabular}{ll}
    \texttt{git clone} & \\
    \texttt{git init} & \\
  \end{tabular}
\end{frame}
%%\begin{frame}[fragile]
%%  \frametitle{Getting git -- Getting Help}
%%
%%  \vspace{1em}
%%  In your terminal at your fingetips \\
%%  \hspace{1em} \texttt{\$ man git} -- Exhaustive, good if you know what you want \\
%%  \hspace{1em} \texttt{\$ man giteveryday} -- Short description of commands \\
%%  \hspace{1em} \texttt{\$ man gittutorial} -- A guide for getting started \\
%%  \hspace{1em} \texttt{\$ man git commit} -- Exhaustive information about the `commit' subcommand\\
%%  \vspace{1em}
%%  From the web, just a GET away \\
%%  \hspace{1em} \url{git-scm.com/doc} -- Links and reference manual \\
%%  \hspace{1em} \url{lithekod.se/gitcheatsheet/} -- LiTHe kods cheat-sheet \\
%%  \hspace{1em} \url{shorturl.at/jpCDQ} -- GitHub's cheat-sheet \\
%%  \hspace{1em} \url{shorturl.at/tuLO3} -- Flowchart by Justin Hileman for messes\\
%%  %TODO meetups
%%
%%\end{frame}

\AtBeginSection[] % Do nothing for \section*
{
\begin{frame}<beamer>
\frametitle{Outline}
\tableofcontents[currentsection]
\end{frame}
}

\section{Basic commands}
\subsection{Getting to know the family}

% When/why to use them
% How to use them
% (Maybe) what you expect to happen

% Edvard
\begin{frame}[fragile]
  \frametitle{Getting to know the family}

  \begin{tabular}{ll}
    \texttt{git add} & I want to keep these! \\
    \texttt{git commit} & I did a thing! \\
    \texttt{git push} & I want to share this! \\ % mention pull
  \end{tabular}

\end{frame}

\begin{frame}
  \frametitle{Getting to know the family -- add}
\end{frame}

\begin{frame}
  \frametitle{Getting to know the family -- commit}
\end{frame}

\begin{frame}
  \frametitle{Getting to know the family -- push}
\end{frame}

\subsection{Commands for information}

% Gustav
\begin{frame}[fragile]
  \frametitle{Commands for information}

  \begin{tabular}{ll}
    \texttt{git status} & What is happening? \\
    \texttt{git diff} & What has changed? \\
    \texttt{git log} & What has happened? \\
    \texttt{git show} & What changed over there? \\
  \end{tabular}

\end{frame}

\begin{frame}
  \frametitle{Commands for information -- status}
\end{frame}

\begin{frame}
  \frametitle{Commands for information -- diff}
\end{frame}

\begin{frame}
  \frametitle{Commands for information -- log}
\end{frame}

\begin{frame}
  \frametitle{Commands for information -- show}
\end{frame}

\subsection{Branching out}

% Erik
\begin{frame}[fragile]
  \frametitle{Branching out}

  \begin{tabular}{ll}
    \texttt{git checkout} & Jump to branch! \\
    \texttt{git branch} & Manage branches! \\
    \texttt{git log} & Find commits to jump to! \\
  \end{tabular}

\end{frame}

\begin{frame}
  \frametitle{Branching out -- checkout}

    \texttt{git checkout} - The sonic screwdriver of git \\[1em]
    Used to:
    \begin{itemize}
        \item Jump to branches
        \item Create branches
        \item Undo changes
    \end{itemize} \\[1em]

    \texttt{git checkout mybranch} \\
    \texttt{git checkout -b newbranch} \\
    \texttt{git checkout -- file.txt}

\end{frame}

\begin{frame}
  \frametitle{Branching out -- branch}

    \texttt{git branch} - Show and manage branches \\[1em]
    Used to:
    \begin{itemize}
        \item<1-> Show which branches exist
        \item<2-> Create branches
        \item<3-> Delete branches
    \end{itemize} \\[1em]

    \pause[4]

    \texttt{git branch} \\
    \texttt{git branch newbranch} \\
    \texttt{git branch -d mergedbranch}
\end{frame}

\begin{frame}
  \frametitle{Branching out -- log (again!)}

    Using \texttt{git log} it is possible to travel in time. \\[1em]

    \only<1>{\texttt{git checkout <hash>}}
    \only<2->{\texttt{git checkout 479ad80ddf...}} \\[3em]

    \pause[3]

    \alert{Watch out! This puts you in 'Detached HEAD state'} \\[1em]

    \pause

    \texttt{"git checkout -"} Puts you back where you were.

\end{frame}

\subsection{Branching in}

\begin{frame}[fragile]
  \frametitle{Branching in}

  \begin{tabular}{ll}
    \texttt{git merge} & I want both this and that! \\
    \texttt{git pull} & I want \emph{your} stuff! \\
  \end{tabular}

\end{frame}

\begin{frame}
  \frametitle{Branching in -- merge} % #1

    Combine two or more branches into one! \\[1em]

    \texttt{git merge featurebranch} \\[3em]

    \pause

    This may result in a so called ``\alert{merge conflict}" \\[1em]

    Again, \emph{don't panic}!
\end{frame}

\begin{frame}
  \frametitle{Branching in -- merge} % #2

    Resolving a merge conflict \\[1em]

    \only<1>{
        \texttt{<<<<<<< HEAD} \\
        \texttt{cool feature} \\
        \texttt{=======} \\
        \texttt{not as cool feature} \\
        \texttt{cool feature} \\
        \texttt{coolest feature} \\
        \texttt{>>>>>>> featurebranch}
    }

    \only<2>{
        \alert{\texttt{<<<<<<< HEAD}} \\
        \alert{\texttt{cool feature}} \\
        \alert{\texttt{=======}} \\
        \texttt{not as cool feature} \\
        \texttt{cool feature} \\
        \texttt{coolest feature} \\
        \alert{\texttt{>>>>>>> featurebranch}}
    }

    \only<3>{
        \texttt{not as cool feature} \\
        \texttt{cool feature} \\
        \texttt{coolest feature} \\[1em]
        Done! Save, add, commit and then push to resolve the conflict!
    }
\end{frame}

\begin{frame}
  \frametitle{Branching in -- pull}
    Get the latest changes from the upstream! \\[1em]

    \texttt{git pull} \\[3em]

    \pause

    This may also result in \alert{merge conflicts}. \\[1em]
\end{frame}

%%% Working with others

% Pause? 5-10 minutes

\section{Working with others}
\subsection{Creating your first Merge Request}

\begin{frame}[fragile]
  \frametitle{Creating your first Merge Request}
\end{frame}

\begin{frame}[fragile]
  \frametitle{Creating your first Merge Request -- Creating your branch}
\end{frame}

\begin{frame}[fragile]
  \frametitle{Creating your first Merge Request -- Doing your thing}
\end{frame}

\begin{frame}[fragile]
  \frametitle{Creating your first Merge Request -- Pushing your branch}
\end{frame}

\begin{frame}[fragile]
  \frametitle{Creating your first Merge Request -- Asking for feedback}
\end{frame}

\subsection{Reading your first Merge Request}

\begin{frame}[fragile]
  \frametitle{Reading your first Merge Request}
\end{frame}

\begin{frame}[fragile]
  \frametitle{Reading your first Merge Request -- Getting their changes}
\end{frame}

\begin{frame}[fragile]
  \frametitle{Reading your first Merge Request -- Testing their changes}
\end{frame}

\begin{frame}[fragile]
  \frametitle{Reading your first Merge Request -- Reviewing their changes}  % maybe two slides
\end{frame}

\subsection{Project management}

\begin{frame}[fragile]
  \frametitle{Project management}
\end{frame}

\begin{frame}[fragile]
  \frametitle{Project management -- Commit messages}
\end{frame}

\begin{frame}[fragile]
  \frametitle{Project management -- Tags}
\end{frame}

\begin{frame}[fragile]
  \frametitle{Project management -- git $\sim$flow$\sim$}
\end{frame}

\begin{frame}[fragile]
  \frametitle{Project management -- CI} % not always needed etc
\end{frame}

\begin{frame}[fragile]
  \frametitle{Project management -- gitignore}
\end{frame}

\section{Advanced commands}

\begin{frame}[fragile]
  \frametitle{Some more commands (bonus round)}
  These commands can be scary and confusing, and hard to `undo'. Worth a google if you
  want to get more use out of git.
  \vspace{1em}
  
  \texttt{git worktree} -- Looking at other branches \\
  \texttt{git cherry-pick} -- Stealing commits \\
  \texttt{git stash} -- WIP commits \\
    
\end{frame}

\end{document}
